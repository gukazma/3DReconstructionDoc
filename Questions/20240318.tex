\subsection{2024/3/18 会议记要}

\begin{itemize}
    \item DOM生成在重建发布里。
    \item 几何来源是图像,那么纹理来源只能是图像。
    \item 如果几何来源来源激光点云和图像,那么纹理贴图就来源于贴图和点云颜色进行结合贴图
    \item 老版本的贴图都是用的本瓦块的信息进行贴图,武大的贴图需要八邻域,周边的simpilymesh跑完之后,在做遮挡检测。
    \item 将三维空间的点投影到照片上,如果顶点算对的话,那么投影点的像素ncc值就应该差不多,那么就需要求ncc梯度。
    \item 泊松融合会将接缝处的颜色匀的比较均匀点。
\end{itemize}