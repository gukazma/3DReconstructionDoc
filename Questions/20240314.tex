\subsection{2024/3/14 会议记要}

\begin{itemize}
    \item QEM算法:计算每个边的中点到相邻面的距离作为代价,如果代价低于一定阈值就将边的两个顶点融合到中点位置。
    \item 网格优化的依据,是三角形投影到图片的分辨率,如果分辨率越高,优化值越高,如果图像降采样,那么优化值就很低,所以图像分辨率必须要够高才行。
    \item 从粗糙到精细的图像分辨率进行优化,每个分辨率迭代20次。
    \item 交叉验证以A为基准算深度,以B为基准算深度。左右一致性检测。
    \item 多分辨率多相关,算每一个像素时,先从较大patch到小的patch,不同感受野计算一个ncc值,如果ncc大于这个阈值,就取更小的感受野。
    \item 
\end{itemize}