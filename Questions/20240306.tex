\subsection{2024/3/6 会议记要}

\begin{itemize}
    \item 几何来源是雷达点云、快速的直接不需要计算密集匹配
    \item  如果是快速的话,$atlas$图就不需要再计算分辨率匹配了。
    \item 深度重置充值$Cach$里的密集点云缓存,结构重置直接构网跳过密集匹配,纹理重置直接到纹理映射
    \item 激光点云的纹理是基于顶点颜色
    \item 几何来源有激光点云的话,基于激光点云和图像进行贴图,会造成纹理出现黑斑的状况,因为激光点云颜色不够精确。
    后来就建议如果有激光点云就只用图像进行贴图,不用激光点云的颜色进行贴图。
    \item 点云是空的,但是图像时有的,就会报错,现在修复。
    \item 如果有$2 \times 1$的情况,即宽大于高,就会出问题,现在的解决方法,就直接按照$1 \times 1$来处理,因为后面的切块操作时按照行来进行操作的。
    \item 大家都跑到纹理贴图就会等周边的8个模型跑完,再算当前块的遮挡关系
    \item 我们的算法是利用ncc进行相对遮挡检测,武大的遮挡检测是对模型和相机的方向进行投影,如果高楼有缺失被分割的瓦块导致缺失,就会导致遮挡检测出问题。
    \item 原来密集匹配和纹理贴图是用的$densepair.bin$,但现在是不同。
    \item 武大会出现电线杆贴到地面的情况,但是我们的是会出现花斑。
    \item 静态点云没有法线,会找到点云的中心然后根据中心计算法线。
    \item 动态点云带轨迹,带雷达行进的轨迹,所以就知道点云的朝向。静态点云不知道朝向。
\end{itemize}